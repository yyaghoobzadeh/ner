
\documentclass[11pt]{article}
\usepackage{coling2016}
\usepackage{times}
\usepackage{url}
\usepackage{latexsym}


\title{Instructions for COLING-2016 Proceedings}

\author{Yadollah Yaghoobzadeh}

\date{}

\begin{document}
\maketitle


\begin{abstract}
I implemented two approaches to solve an NER problem in news domain:
one deep learning based and one based on CRF. 

\end{abstract}
%
\section{Introduction}
In this report, I first briefly explain the problem and then 
two proposed solutions. 


\section{Problem definition}

Based on the description, the aim of this task is to extract ``entities'' from 
sentences and categorize them into several labels.
A  training dataset is provided as part of the task. 
The dataset consists of sentences or phrases, 
in which each word is annotated with IOB and four entity type tags. 
The types are: 
NEWSTYPE, PROVIDER, SECTION and KEYWORDs.
The sentences seem to be the user commands for a dialog system
requesting some kind of news (e.g., breaking, spotlight) 
from some providers (e.g., star democrat, douglas country news) 
about some keywords (e.g., obama, ibm) 
in different sections 
(e.g., wellness, health).


\section{Solving problem}
The training dataset consists of 1,000 sentences.
To estimate the hyperparameters, we divide
these sentence into 658 / 342 as train/dev.
(Since the distribution of data seems to be similar and small improvements
are not desirable, we did not do cross validation.)
We propose two methods to solve the problem:
one deep learning based method and one 
based on linear CRF. 


\subsection{Deep Learning Based model: LSTM-NER}
\textbf{Model description.}
For this part, we use
long short term memory (LSTM) network
to extract character and word level features of sentence words
automatically. 
My model is similar to the model in \cite{tagger}, 
so we used their implementation\footnote{https://github.com/glample/tagger} and modified it.
In more details, 
one LSTM is going over sequence of characters in each word resulting in a 
character-level representation of words. 
Another LSTM is going over sequence of words, in which words are represented 
by the concatenation of word and character level embeddings. 
The word embeddings are either initialized randomly or by pretrained word embeddings. 
The hidden states of LSTM are considered as contextual representation of
words and fed into a softmax layer to output tags probabilities for each word. 
We can also use a backward LSTM resulting in Bi-directional LSTM.
For more details, see \newcite{tagger}.

\textbf{Transfer learning.}
The training dataset is quite small (1,000 sentences), 
and the variation of input data is large to be captured well.
Therefore, we propose to do transfer learning. 

The basic yet effective transfer learning that can be used is to 
initialize the word embedding layer with pretrained 
embeddings. 
Word embeddings are generic representation of words trained using unsupervised methods 
like word2vec, glove, etc. 

Apart from word embedding layer, we can transfer the knowledge from 
another NER dataset to this new one. 
As far as we know, there is no publicly available dataset with the same tags 
as our training data, but there is still dependencies among these tags
and the more standard ones in other datasets like CoNLL. 
Therefore, we propose a two step training regime. 
First, we train our LSTM-NER model on CoNLL 2003 dataset (tags are: ORG, LOC, PER, MISC).
Then, we modify the LSTM-NER (lets call it LSTM-NER)
and add another softmax layer to the output of LSTM-NER
to output the new tag-set probabilities. 
We initialize the LSTM-NER-TWO with the parameters 
from LSTM-NER.
We hypothesize that especially for KEYWORDS and PROVIDER tags in the training data,
the training on CoNLL data can really be helpful since 
these two are more related to the tags of CoNLL.

\subsection{Feature based model: CRF-NER}
For this part, we use a CRF based model using hand made features. 
For each word, we extract features from itself and surrounding words.
Since in the datasets, all letters are lowercase, one of the 
most important feature of named entities is missing. 
So, the CRF should at least use features from POS tags, letters in the word, the word itself.
We found these features to give reasonable results on our test set. 
In more details, features for each position of words in the sentence are:
\begin{itemize}
\item 
position word 
\item 
last trigram of word
\item 
last bigram of word
\item 
POS tag
\item 
word and POS tag of the window size of 2 around the position.
\end{itemize}

We also tried adding more features, but it hurted the performance rather than helping
probably due to the small size of the training data. 

\section{Experiments}
\subsection{Implementations and setups}
\textbf{LSTM-NER.}
The implementation is in Python 2.7 with
installed Theano 0.8.2, numpy 1.10.4., Scipy 0.17.0. 

We use Glove pretrained word embeddings with 50 dimensions to initialize the 
word embedding layer.
We first train LSTM-NER on CoNLL 2003 dataset. 
We use BiLSTM for words and LSTM for characters. 

Then, we retrain the model on the training dataset of this task using 
the same parameters. 
We take the best model on dev and compare the results with different parametrization. 

\textbf{CRF-NER}
The implementation is in Python 2.7 with
installed NLTK 3.2, numpy 1.10.4., Scipy 0.17.0. 
Pycrfsuite \footnote{\url{https://github.com/tpeng/python-crfsuite}}.

\subsection{Evaluation}
As we already mentioned, we divided
the three sentence into 658 / 342 as train/dev.
In Table 1, we show the performance of our best models on the dev data.
\begin{table}[ht]
\begin{center}
\begin{tabular}{l|c|c|c|c}
\hline 
& prec & rec & F1 & Acc\\ 
\hline 
CRF-NER  & 93.17 & 92.31 & 92.74 & 97.43\\ 
\hline 
LSTM-NER & 95.96 & 96.94 & 96.45 & 98.95\\ 

\hline 
\end{tabular} 
\caption{Overall results on the test set. }
\end{center}
\label{perf}
\end{table}

\textbf{Feature analysis.}

\begin{table}
\begin{center}
\begin{tabular}{l|c}
\hline 
added feature & F1 \\ 
\hline 
word     & 85.66 \\
+trigram & 87.45 \\
+bigram  & 87.75 \\
+POS  & 88.28 \\
+ surrounding words & 92.4 \\
+ POS of surrounding words & 92.59
\end{tabular} 
\caption{CRF-NER feature analysis.}
\end{center}
\label{perf}
\end{table}



\bibliography{ref.bib}
\bibliographystyle{acl}


\end{document}
