\documentclass{llncs}

\usepackage[utf8]{inputenc}
\usepackage[T1]{fontenc}
\usepackage[brazilian]{babel}
\usepackage{csquotes}
\usepackage{hyphenat}
\hyphenation{pro-ble-ma}

\usepackage{amssymb}
\usepackage{amsmath}
\usepackage{array}

% Better handle of numeric things.
% Configured to French style (same 
% as Brazilian conventions)
\usepackage[locale=FR]{siunitx}

\usepackage{graphicx}
\graphicspath{ {images/} }

\usepackage{listings}

\begin{document}

\title{Named Entity Recognition in News domain: Report}
%

\author{Yadollah Yaghoobzadeh}
%
%
%%%% list of authors for the TOC (use if author list has to be modified)
%

\maketitle              % typeset the title of the contribution

\begin{abstract}
I implemented two approaches to solve an NER problem in news domain:
one deep learning based and one based on CRF. 

\end{abstract}
%
\section{Deep Learning Based NER}
%
\subsection{Model description}
For this part, I used 
long short term memory (LSTM) network
to extract character and word level features
automatically. 
My model is similar to the model in \cite{tagger}, 
so I used their implementation\footnote{https://github.com/glample/tagger} and modified it.

\subsection{Training Model}
My implementation needs Theano 0.8.2, Python 2.7.



%



\end{document}
